% Develop a LaTeX script to demonstrate the presentation of Numbered theorems, definitions, corollaries,
% and lemmas in the document

\documentclass{article}
\usepackage[english]{babel}
% \usepackage{gensymb} % For the degree symbol

\newtheorem{theorem}{Theorem}[section]
\newtheorem{corollary}[theorem]{Corollary}
\newtheorem{lemma}[theorem]{Lemma}

\begin{document}

\section{Introduction}
Theorems can be easily defined:
\begin{theorem}
Let \(f\) be a function whose derivative exists in every point, then \(f\) is a continuous function.
\end{theorem}

\begin{theorem}[Pythagorean theorem]
\label{pythagorean}
This is a theorem about right triangles and can be summarized in the next equation:
\[ x^2 + y^2 = z^2 \]
\end{theorem}

And a sequence of Theorem \ref{pythagorean} is the statement in the next corollary.

\begin{corollary}
There's no right triangle whose sides measure 3 cm, 4 cm, and 6 cm.
\end{corollary}

You can reference theorems such as Theorem \ref{pythagorean} when a label is assigned.

\begin{lemma}
Given two line segments whose lengths are \(a\) and \(b\) respectively, there is a real number \(r\) such that \(b = ra\).
\end{lemma}

\end{document}