\documentclass{report}
\usepackage{graphicx}
\title{Airlines Management System}
\author{Rajiv}
\date{\today}
\begin{document}
\maketitle
\tableofcontents
\chapter{Introduction}
This project on Airline Management System is the automation of registration process of airline system. The system is able to provide much information like passenger’s information, criminal’s, list of all passengers etc. The system also allows us to add records when a passenger reserves a ticket. For data storage and retrieval we use the file-handling facility of C Language. It enables us to add any number of records in our database. But for intrinsic nature of file handling, the retrieval process is slow when we
Search a particular record in the database, because record  is searched sequentially.
\section{Background}
The project named “Airline Management System” is written in Turbo C++ IDE3.0, mainly because of it’s suitability for this type of application. Its user friendly nature and in-built documentation, complication, error detection, binding facilities and interaction with other software packages make it most powerful tool for software development. Moreover .Turbo C++ consists of all the technologies that help in creating and running robust, scalable and distributed packages.C++ is a general-purpose object-oriented programming language, and is intended to be an improved C with object capabilities
Assistance is provided to the user at each and every step so that no problem is faced during using it. Further the details of every process and the user manuals attached in the report make it very easy to understand. Every possible care has been taken to make the software and the report clear, simple and error free which makes it so special and one of its kind.
\chapter{objective}
    • To provide some amount of automation in  airlines mangement.\\
    • To help airlines system in making their business more efficient.\\
    • An added attraction for their potential customers.\\
    • It will also show the attitute of the management that they are aware to the newly
      introduced technology and ready to adopt them.
\chapter{Requirement Specifications}
    Frontend:\\

    User interface for booking, seat selection, and check-in.\\
    Responsive design for mobile and desktop.\\
    Integration with payment gateways and external APIs.\\

    Backend:\\

    Server-side logic for authentication, flight management, and inventory.\\
    Database management for passenger data, flight schedules, and transactions.\\
\chapter{Schema diagram}
\begin{figure}[h]
\centering
\caption{Schema diagram}
\label{fig:schema diagram}
\includegraphics[width=1.0\textwidth]{schema.png}
\end{figure}
\chapter{Implementation}
\begin{figure}[h]
\centering
\includegraphics[width=1.0\textwidth]{imp.png}
\end{figure}
\chapter{Conclusion}
In conclusion, the implementation of an effective Airlines Management System (AMS) is crucial for optimizing operational efficiency, enhancing customer satisfaction, and ensuring financial sustainability within the airline industry. Our report highlights the multifaceted benefits that a robust AMS can provide, including streamlined scheduling, improved resource allocation, and real-time data access, which collectively contribute to a more responsive and adaptive operational framework.\cite{lopez2010implementation}
\bibliographystyle{plain}
\bibliography{reference12}
\end{document}
